\documentclass{stylesheet}

\usepackage{color}
\usepackage{url}
\usepackage{enumerate}
\usepackage[english]{babel}%spell check
\usepackage[colorinlistoftodos]{todonotes}%todo
\usepackage{etoolbox}%Remove the ugly copyright from the template :-)
\makeatletter%Remove the ugly copyright from the template :-)
\patchcmd{\maketitle}{\@copyrightspace}{}{}{}%Remove the ugly copyright from the template :-)
\makeatother%Remove the ugly copyright from the template :-)

\usepackage[nameinlink,capitalise]{cleveref}

\begin{document}
\title{Cloud Computing}

\numberofauthors{5} 
\author{
% 1st. author
\alignauthor
Christos Froussios\\
	\affaddr{4322754}\\
	\email{C.Froussios @student.tudelft.nl}
% 2nd. author
\alignauthor
Richard van Heest\\
	\affaddr{4086570}\\
	\email{A.W.M.vanHeest @student.tudelft.nl}
\and
% Alex
\alignauthor
Alexandru Iosup\\
	\affaddr{Course instructor}
	\email{A.Iosup@tudelft.nl}
% Dick
\alignauthor
Dick Epema\\
	\affaddr{Course instructor}
	\email{D.H.J.Epema@tudelft.nl}
% Alexey
\alignauthor
Alexey Ilyushkin\\
	\affaddr{Teaching Assistant}
	\email{A.S.Ilyushkin@tudelft.nl}
}

\maketitle

\begin{abstract}
In this report we describe our findings of 1 month of research in the area of cloud computing. We build a small prototype of a IaaS-based application that does image processing using the Amazon Web Services (AWS) EC2 cloud platform. The main features of this prototype are automation, scalability, load balancing, reliability and monitoring. We measured the performance of the system by applying metrics like \ldots
\end{abstract}

\section{Introduction}
Cloud computing has gained an increasing interest in the last couple of years. In contrast to the era of grid computing, not only the academic world is interested in this new way of performing large scale computations, but also companies do so. The most logic explanation for this shift of interest is that computer grids are expensive to invest in for a company which may not use the required hardware constantly, whereas in cloud computing that company would lease the required machines, scale up or down if appropriate and only get charged for the hours between leasing and releasing.

We distinguish three types of cloud computing:
\begin{description}
	\item[Software as a Service (SaaS)] applications such as Google Drive, Dropbox, Gmail, Yahoo mail and Facebook.
	\item[Platform as a Service (PaaS)] the computing platforms which typically include operating systems, programming language execution environments, databases and web servers. Examples of this are Apache Hadoop, Google App Engine, Windows Azure and Amazon Web Services Elastic Beanstalk.
	\item[Infrastructure as a Service (IaaS)] the computing infrastructure, physical or (more often) virtual machines and other resources like virtual-machine disk image libraries, block and file-based storage, firewalls, load balancers, IP addresses and virtual local area networks. The most prominent products in this category are Windows Azure, Google Compute Engine and Amazon EC2.
\end{description}

In this report we will focus on IaaS cloud computing especially, as requested to investigate by WantCloud BV. We will look into the advantages and disadvantages of IaaS by implementing a small prototype for a potential future system. The main features of this prototype can be summarized as:
\begin{description}
	\item[Automation] working as much as possible independently from any human interaction
	\item[Elasticity (auto-scaling)] leasing and releasing machines from a resource pool as workloads change over time
	\item[Performance (load balancing)] allocating workloads to machines from the resource pool in such a way that the machines are used as effective as possible.
	\item[Reliability] building in a fair amount of fault tolerance
	\item[Monitoring] observing and recording the status of the system as well as measuring the performance of the system and its components by applying various metrics.
	\item[TODO] \textbf{\textit{\underline{additional features???}}}
\end{description}

The prototype to be implemented will be an application that receives pictures from an external source (for example a web form where users can submit their pictures), performs some operations on them and returns the processed pictures. These operations may vary from lightweight operations such as a flip of the picture to applying a Fourier transformation.

The application is set up in such a way that all pictures are received at a head node, which allocates them to one of the worker node in its resource pool and leases new machines or releases idle machines when appropriate. For fault tolerance reasons, the allocated picture is temporarily stored on the head node until its processed counterpart is received back in the head node, such that in the case of a failing worker node, the picture can be reallocated.

\textcolor{red}{TODO allocation policies, lease/release policies, communication, monitoring, head-node-never-fails-assumption?, \ldots}

This report is structured as follows: first we will discuss the background of the application in \cref{sec:background}. What does it do in detail? What are its requirements and features? This is followed by the system design in \cref{sec:design}, where we focus on the internal workings of the application as well as several policies for both the processes of allocation and leasing/releasing nodes. We evaluate the application in \cref{sec:experiments} by applying several metrics. Finally we close the report with a discussion (\cref{sec:discussion}) and the conclusion (\cref{sec:conclusion}). In \cref{app:time} an overview of spent time is provided.

\textcolor{red}{TODO Problem description, existing systems and/or tools, system to be implemented, structure of the rest of the paper}

\section{Background of the application}
\label{sec:background}
Description of the application and its requirements

\section{System Design}
\label{sec:design}
\ldots
\subsection{Resource Management}
\label{subsec:resourceManagement}
Description of the design of the system, inlcuding the main features required by the assignment

\subsection{System policies}
\label{subsec:policies}
Description of the supported (future work) and used policies.

\subsection{Additional System Features}
\label{subsec:additionalFeatures}
\ldots
\subsubsection{Feature 1}
\ldots
\subsubsection{Feature 2}
\ldots

\section{Experiments}
\label{sec:experiments}
\ldots
\subsection{Experimental setup}
\label{subsec:setup}
Working environment (EC2), general workload and monitoring tools and other libraries.

\subsection{Experimental results}
\label{subsec:results}
Description of the experiments conducted to analyze each system feature, analyze them and describe the workload, present the operation of the system and analyze the results.

\section{Discussion}
\label{sec:discussion}
main findings, tradeoffs inherent in the design of cloud-computing-based applications.

\section{Conclusion}
\label{sec:conclusion}
\ldots

\appendix
\section{Time sheet}
\label{app:time}

\bibliographystyle{abbrv}
\bibliography{references}
\end{document}