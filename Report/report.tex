\documentclass{stylesheet}

\usepackage{color}
\usepackage{url}
\usepackage{enumerate}
\usepackage[english]{babel}%spell check
\usepackage[colorinlistoftodos]{todonotes}%todo
\usepackage{etoolbox}%Remove the ugly copyright from the template :-)
\makeatletter%Remove the ugly copyright from the template :-)
\patchcmd{\maketitle}{\@copyrightspace}{}{}{}%Remove the ugly copyright from the template :-)
\makeatother%Remove the ugly copyright from the template :-)

\usepackage[nameinlink,capitalise]{cleveref}

\begin{document}
\title{Cloud Computing}

\numberofauthors{5} 
\author{
% 1st. author
\alignauthor
Christos Froussios\\
	\affaddr{4322754}\\
	\email{C.Froussios @student.tudelft.nl}
% 2nd. author
\alignauthor
Richard van Heest\\
	\affaddr{4086570}\\
	\email{A.W.M.vanHeest @student.tudelft.nl}
\and
% Alex
\alignauthor
Alexandru Iosup\\
	\affaddr{Course instructor}
	\email{A.Iosup@tudelft.nl}
% Dick
\alignauthor
Dick Epema\\
	\affaddr{Course instructor}
	\email{D.H.J.Epema@tudelft.nl}
% Alexey
\alignauthor
Alexey Ilyushkin\\
	\affaddr{Teaching Assistant}
	\email{A.S.Ilyushkin@tudelft.nl}
}

\maketitle

\begin{abstract}
In this report we describe our findings of 1 month of research in the area of cloud computing. We build a small prototype of a IaaS-based application that does image processing using the Amazon Web Services (AWS) EC2 cloud platform. The main features of this prototype are automation, scalability, load balancing, reliability and monitoring. We measured the performance of the system by applying metrics like \textcolor{red}{TODO \ldots}
\end{abstract}

\section{Introduction}
\textcolor{red}{+ title + abstract (150 words) = 1 page}

Cloud computing has gained an increasing interest in the last couple of years. In contrast to the era of grid computing, not only the academic world is interested in this new way of performing large scale computations, but also companies do so. The most logic explanation for this shift of interest is that computer grids are expensive to invest in for a company which may not use the required hardware constantly, whereas in cloud computing that company would lease the required machines, scale up or down if appropriate and only get charged for the hours between leasing and releasing.

We distinguish three types of cloud computing:
\begin{description}
	\item[Software as a Service (SaaS)] applications such as Google Drive, Dropbox, Gmail, Yahoo mail and Facebook.
	\item[Platform as a Service (PaaS)] the computing platforms which typically include operating systems, programming language execution environments, databases and web servers. Examples of this are Apache Hadoop, Google App Engine, Windows Azure and Amazon Web Services Elastic Beanstalk.
	\item[Infrastructure as a Service (IaaS)] the computing infrastructure, physical or (more often) virtual machines and other resources like virtual-machine disk image libraries, block and file-based storage, firewalls, load balancers, IP addresses and virtual local area networks. The most prominent products in this category are Windows Azure, Google Compute Engine and Amazon EC2.
\end{description}

In this report we will focus on IaaS cloud computing especially, as requested to investigate by WantCloud BV. We will look into the advantages and disadvantages of IaaS by implementing a small prototype for a potential future system. The main features of this prototype can be summarized as:
\begin{description}
	\item[Automation] working as much as possible independently from any human interaction
	\item[Elasticity (auto-scaling)] leasing and releasing machines from a resource pool as workloads change over time
	\item[Performance (load balancing)] allocating workloads to machines from the resource pool in such a way that the machines are used as effective as possible.
	\item[Reliability] building in a fair amount of fault tolerance
	\item[Monitoring] observing and recording the status of the system as well as measuring the performance of the system and its components by applying various metrics.
	\item[TODO] \textbf{\textit{\underline{additional features???}}}
\end{description}

The prototype to be implemented will be an application that receives pictures from an external source (for example a web form where users can submit their pictures), performs some operations on them and returns the processed pictures. These operations may vary from lightweight operations such as a flip of the picture to applying a Fourier transformation.

The application is set up in such a way that all pictures are received at a head node, which allocates them to one of the worker node in its resource pool and leases new machines or releases idle machines when appropriate. For fault tolerance reasons, the allocated picture is temporarily stored on the head node until its processed counterpart is received back in the head node, such that in the case of a failing worker node, the picture can be reallocated.

\textcolor{red}{TODO allocation policies, lease/release policies, communication, monitoring, head-node-never-fails-assumption?, \ldots}

This report is structured as follows: first we will discuss the background of the application in \cref{sec:background}. What does it do in detail? What are its requirements and features? This is followed by the system design in \cref{sec:design}, where we focus on the internal workings of the application as well as several policies for both the processes of allocation and leasing/releasing nodes. We evaluate the application in \cref{sec:experiments} by applying several metrics. Finally we close the report with a discussion (\cref{sec:discussion}) and the conclusion (\cref{sec:conclusion}). In \cref{app:time} an overview of spent time is provided.

\textcolor{red}{TODO Problem description, existing systems and/or tools, system to be implemented, structure of the rest of the paper}

\section{Background of the application}
\label{sec:background}
\textcolor{red}{0.5 page: describe application (1 paragraph) + requirements (1-3 paragraphs)}

The application we build over the course of several weeks is a prototype of a more generic cloud centered solution where tasks come in at the head node and are distributed over a number of worker nodes. In this particular application we choose to take image processing as the workload. Images that are received by an arbitrary worker are modified by applying a number of operations, combining them afterwards. Notice that this approach has a high potential for the task being split into subtasks and make it into a MapReduce process. However, to make things not overly complicated in this prototype, we consider one image as one \textit{undividable} task, which is executed on a single node.

The requirements of our application can be split up into 5 tiers. First of all, the application needs to run without human intervention as much as possible. We fulfilled this requirement by only having to start the head node application and providing images over the course of the runtime. Besides these 2 acts, the head node automatically decides when to lease and release worker nodes, to which worker an incoming task is sent and to where a completed task needs to be returned. On the other hand, the worker nodes automatically receive tasks, process them as soon as they possible can and make sure they send back the result to the head node.

A second tier holds the elasticity of the system, which is determined by the head node. Elasticity in the context of cloud computing describes to which degree the system is able to grow and shrink its number of resources. In this case, these resources are the worker nodes, which can be leased or released. When the system gets overloaded with tasks, it might be a good idea to lease more workers. On the other hand, when there are hardly any tasks in the system, it is useless to have a large number of idle workers, since they are being payed for all that time. Ideally you want to predict when the system will have a peak moment in the number of tasks and when it will get more quiet. For this, machine learning is a nice path to follow. This however is not part of this project.

Besides elasticity, load balancing is another topic of interest in cloud computing. When the head node receives a task, it needs to decide where it should be executed. Does it prefer certain workers, will it be assigned randomly or is there a better way to schedule a task. In this the head node might look at the size of the task, the length of the queue of each available worker or any other measurable property of the task and the workers.

The fourth tier is the reliability of the system. As the application is mostly autonomous, it also needs to be able to recover from failure. A failure might be losing connection between the head node and one or more workers, crashing the head or a worker, IO failure, etc. For this project we make the assumption that the head node never crashes and \textcolor{red}{complete this when we implemented this}.

Finally, a system isn't a good system without a monitoring module. To make decisions in the other tiers (especially elasticity and load balancing), we need to supply it with proper data, for example the length of each worker's queue, the number of incoming tasks in the system, etc. Also for the research part of this project the monitoring module is used. This is used to gain the results in \cref{sec:experiments}.

\section{System Design}
\label{sec:design}
\textcolor{red}{1.5 pages}
\subsection{Resource Management}
\label{subsec:resourceManagement}
\textcolor{red}{Description of the design of the system + provisioning, allocation, reliability and monitoring}

\subsection{System policies}
\label{subsec:policies}
\textcolor{red}{Description of the supported (future work) and used policies.}

\subsection{Additional System Features}
\label{subsec:additionalFeatures}
\textcolor{red}{Additional features}
\ldots
\subsubsection{Feature 1}
\ldots
\subsubsection{Feature 2}
\ldots

\section{Experiments}
\label{sec:experiments}
\textcolor{red}{1.5 pages}
\subsection{Experimental setup}
\label{subsec:setup}
\textcolor{red}{Working environment (EC2), general workload and monitoring tools and other libraries}

\subsection{Experimental results}
\label{subsec:results}
\textcolor{red}{Description of the experiments conducted to analyze each system feature, analyze them and describe the workload, present the operation of the system and analyze the results. See assignment}

\section{Discussion}
\label{sec:discussion}
\textcolor{red}{1 page - main findings, tradeoffs inherent in the design of cloud-computing-based applications.}

\section{Conclusion}
\label{sec:conclusion}
\textcolor{red}{\ldots}

\bibliographystyle{abbrv}
\bibliography{references}

\appendix
\section{Time sheet}
\label{app:time}
\textcolor{red}{table with time spend}
\end{document}